%Version 2.1 April 2023
% See section 11 of the User Manual for version history
%
%%%%%%%%%%%%%%%%%%%%%%%%%%%%%%%%%%%%%%%%%%%%%%%%%%%%%%%%%%%%%%%%%%%%%%
%%                                                                 %%
%% Please do not use \input{...} to include other tex files.       %%
%% Submit your LaTeX manuscript as one .tex document.              %%
%%                                                                 %%
%% All additional figures and files should be attached             %%
%% separately and not embedded in the \TeX\ document itself.       %%
%%                                                                 %%
%%%%%%%%%%%%%%%%%%%%%%%%%%%%%%%%%%%%%%%%%%%%%%%%%%%%%%%%%%%%%%%%%%%%%

\documentclass[sn-apa,pdflatex]{sn-jnl}

%%%% Standard Packages
%%<additional latex packages if required can be included here>

\usepackage{graphicx}%
\usepackage{multirow}%
\usepackage{amsmath,amssymb,amsfonts}%
\usepackage{amsthm}%
\usepackage{mathrsfs}%
\usepackage[title]{appendix}%
\usepackage{xcolor}%
\usepackage{textcomp}%
\usepackage{manyfoot}%
\usepackage{booktabs}%
\usepackage{algorithm}%
\usepackage{algorithmicx}%
\usepackage{algpseudocode}%
\usepackage{listings}%
%%%%

%%%%%=============================================================================%%%%
%%%%  Remarks: This template is provided to aid authors with the preparation
%%%%  of original research articles intended for submission to journals published
%%%%  by Springer Nature. The guidance has been prepared in partnership with
%%%%  production teams to conform to Springer Nature technical requirements.
%%%%  Editorial and presentation requirements differ among journal portfolios and
%%%%  research disciplines. You may find sections in this template are irrelevant
%%%%  to your work and are empowered to omit any such section if allowed by the
%%%%  journal you intend to submit to. The submission guidelines and policies
%%%%  of the journal take precedence. A detailed User Manual is available in the
%%%%  template package for technical guidance.
%%%%%=============================================================================%%%%

%% Per the spinger doc, new theorem styles can be included using built in style, 
%% but it seems the don't work so commented below
%\theoremstyle{thmstyleone}%
\newtheorem{theorem}{Theorem}%  meant for continuous numbers
%%\newtheorem{theorem}{Theorem}[section]% meant for sectionwise numbers
%% optional argument [theorem] produces theorem numbering sequence instead of independent numbers for Proposition
\newtheorem{proposition}[theorem]{Proposition}%
%%\newtheorem{proposition}{Proposition}% to get separate numbers for theorem and proposition etc.
%% \theoremstyle{thmstyletwo}%
\theoremstyle{remark}
\newtheorem{example}{Example}%
\newtheorem{remark}{Remark}%
%% \theoremstyle{thmstylethree}%
\theoremstyle{definition}
\newtheorem{definition}{Definition}%



\raggedbottom




% tightlist command for lists without linebreak
\providecommand{\tightlist}{%
  \setlength{\itemsep}{0pt}\setlength{\parskip}{0pt}}





\begin{document}


\title[Article Title runing]{Graphs as Not Only Relational Databases for
Behavior Science}

%%=============================================================%%
%% Prefix	-> \pfx{Dr}
%% GivenName	-> \fnm{Joergen W.}
%% Particle	-> \spfx{van der} -> surname prefix
%% FamilyName	-> \sur{Ploeg}
%% Suffix	-> \sfx{IV}
%% NatureName	-> \tanm{Poet Laureate} -> Title after name
%% Degrees	-> \dgr{MSc, PhD}
%% \author*[1,2]{\pfx{Dr} \fnm{Joergen W.} \spfx{van der} \sur{Ploeg} \sfx{IV} \tanm{Poet Laureate}
%%                 \dgr{MSc, PhD}}\email{iauthor@gmail.com}
%%=============================================================%%

\author*[1,2]{\fnm{Juan} \spfx{C.} \sur{Correa} }\email{\href{mailto:juan.correa@ibero.mx}{\nolinkurl{juan.correa@ibero.mx}}}



  \affil*[1]{\orgdiv{Departmento de Estudios
Empresariales}, \orgname{Universidad
Iberoamericana}, \orgaddress{\city{Mexico
City}, \country{Mexico}, \postcode{1219}}}
  \affil*[2]{\orgdiv{Research and Development Unit}, \orgname{Critical
Centrality Institute}}

\abstract{Behavior scientists have multiple options for managing
research data. While relational databases offer robust tools for storage
and analysis, they impose structural constraints that limit the
representation of complex behavioral phenomena. This article argues that
graph databases provide a rigorous and conceptually richer alternative,
enabling the modeling of human behavior as an interconnected system
rather than a set of isolated variables. Beyond a technological shift,
adopting graph-based approaches invites a paradigm change in behavior
science---one that embraces complexity, dynamic relationships, and
multi-level contingencies. Practical implications are illustrated
through examples from clinical, consumer, and industrial/organizational
psychology.}

\keywords{key, dictionary, word}


\pacs[JEL Classification]{D8, H51}
\pacs[MSC Classification]{35A01, 65L10}

\maketitle

\section{Introduction}\label{sec1}

In a recent paper, \citet{soto2025} introduced relational databases for
behavior science and used real-world examples to illustrate how
relational databases have been used by behavior scientists. Even though
relational databases represent the dominant paradigm inside and outside
academic research settings, other paradigms are gaining traction. The
so-called ``Not Only Relational Database'' encompasses a series of
database management systems that use single data structures to hold
information. There are several instances of single data structures such
as lists, key-value pairs, wide columns, documents, matrices, or graphs.

In this article, I examine graphs as a different paradigm from the
traditional paradigm of relational databases, where nodes and edges
(instead of tables and joins) represent the basic elements of any
behavior that can be represented as a network or complex system.
Networks have a long history in mathematics as ``\emph{graph theory}''
\citep{Estrada2011}. In sociology and social sciences, graph theory is
known as ``social network analysis'' \citep{Wasserman1994}, and
psychologists have leveraged this framework to analyze the structure of
psychopathology \citep{borsboom2013}, estimate the correct number of
dimensions in psychological and educational instruments
\citep{golino2017}, or understand the measurement of organizational
climate \citep{menezes2021}.

\section{Network as a collection nodes and edges}\label{sec2}

One of the easiest way to grasp the idea of a network is by looking its
visual representation (see Figure \hyperref[fig:F1]{1}). According to
\citet{Estrada2011}, a network is a collection of points (called nodes)
joined together in pairs by lines (called arcs or edges). Scientists can
model different kinds of networks from physical networks (e.g., flights
between airports) to biological networks (e.g., protein-protein
interactions), and social networks (e.g., who follows whom in LinkedIn
or X). In this context, the term ``social network'' should not be
confused with online platforms such as Facebook or Instagram, as they
are technological implementations that do not necessarily represent all
aspects of social networks as an academic discipline.

\begin{figure}[H]

{\centering \includegraphics[width=227px,]{../F1} 

}

\caption{A visual representation of four types of simple networks: non-directed unweighted network (upper left), non-directed weighted network (upper right), di- rected unweighted network (bottom left), and directed weighted network (bottom right).}\label{fig:F1}
\end{figure}

Graphs offer fundamental concepts for understanding how entities (nodes)
and their relationships (edges) form interconnected structures. Network
structures underpin a wide range of behavioral phenomena ---from disease
transmission and social clustering to the spread of information and
misinformation. As we have mentioned above, the recognition of these
patterns requires tools that go beyond the rigid tabular constraints of
relational databases. As the concepts of graphs are thoroughly covered
in introductory texts \citep{Newman2010}, these will not be revisited
here. Instead, this article aims to illustrate how graph-based databases
can enrich the methodological toolbox of behavior scientists, enabling
analyses that embrace complexity, dynamic relationships, and multi-level
contingencies \citep{Hunger2021}.

\section{Graph databases: A gentle
introduction}\label{graph-databases-a-gentle-introduction}

A configuration is a specific pattern of connections between nodes. The
most elementary configuration is a dyad (two connected nodes). Other
common configurations include triangles (three nodes, all connected to
each other) and k-stars (a central node connected to k other nodes).
These configurations, along with others, make up the larger connected
subgraphs of a network, known as components. A component is a set of
nodes where a path exists between every pair of nodes. A component is a
maximal connected subgraph, where all nodes within it are reachable from
each other, and it is not connected to any other nodes. A component can
be as simple as a long line of connected nodes with no triangles or
k-stars. A path is a fundamental concept that describes a sequence of
connections between nodes. It is the route one can follow to travel from
one node to another. Paths are essential for understanding properties
like the shortest distance between any pair of nodes. A special type of
path is a cycle, which is a closed path with no repeated nodes or edges
other than the start and end nodes. A cycle has a minimum of three links
(e.g., a triangle). From the network science perspective, geometric
structures like squares, pentagons, or decagons can all be seen as
cycles. The four types of configurations in a network are depicted in
Figure \hyperref[fig:F2]{2}.

\begin{figure}[H]

{\centering \includegraphics[width=227px,]{../F2} 

}

\caption{Basic configurations in a network: dyads (top left), k-stars (top right), triples and triangles (bottom left), and components (bottom right).}\label{fig:F2}
\end{figure}

In the analysis of social networks, it is very important to distinguish
between connected and disconnected networks \citep{Wasserman1994}. When
a network is connected there is a path between every pair of nodes; that
is, you can reach any node from any other node. In a disconnected
network some paths do not exist (i.e., some nodes are isolated and not
reachable). The nodes in a disconnected network can be partitioned into
two or more subsets in which there are no paths between the nodes in
different subsets. The connected sub-networks in a network are called
components. A component is a sub-network in which there is a path
between all pairs of nodes and there is no path between a node in a
component and any node other node in the other component.

\subsection{Nodal Centrality}\label{nodal-centrality}

Nodal centrality relates to the amount of connections a given node has,
and can be computed in different ways \citep{oldham2019}. Nodal degree
is the most basic way of estimating a node centrality. It requires
counting the number of lines incident with it. For example, the nodal
degree of the central node in the 3-star network depicted in the top
right of Figure \hyperref[fig:F2]{2} is three.

Note that nodal centrality is a property of every single node in a
network, and when the analysis of a social network focuses on nodes
centrality, this estimation is done to every single node in the network.
The analysis of nodes centrality facilitates enables the differentiation
of most important nodes from most peripheral nodes and organize nodes in
order of importance or number of connections in the network.

\section{Applications of Network
Modeling}\label{applications-of-network-modeling}

Network modeling has been used by psychologists and behavioral
scientists in several ways that fit into the professional goals of
well-established APA's divisions such as quantitative and qualitative
methods (division 5), clinical psychology (division 12), and industrial
and organizational psychology (division 14). The following subsections
summarize how network modeling has been applied in these subdisciplines.

\subsection{Network modeling as a quantitative method}\label{sec3}

\citet{golino2017} illustrated the principles by which network modeling
can be used as a framework for estimating the correct number of
dimensions in psychological and educational instruments. While
traditional techniques such as ``parallel analysis'' and ``minimum
average partial'' are widely utilized, they often underestimate the
number of factors in scenarios involving high correlations between
latent variables, small sample sizes, or few indicators per factor.
Conversely, the widely used Kaiser-Guttman rule is frequently criticized
for overestimating the number of factors, particularly as sample sizes
and the number of items increase.

In response to these limitations, ``exploratory graph analysis'' has
been developed as a new approach derived from the field of network
psychometrics \citep{epskamp2018}. This framework utilizes Markov random
fields to model the interaction between random variables as a network of
nodes (variables) and edges (direct relationships). Specifically, EGA
employs the \textbf{Gaussian Graphical Model (GGM)}, which models the
multivariate distribution through the inverse covariance matrix. When
these elements are standardized, they represent \textbf{partial
correlation coefficients}. A fundamental tenet of this approach is the
``Clusters in network = latent variables'' rule: if a latent variable
model represents the true causal structure, its indicators will manifest
as strongly connected clusters (or cliques) within the network because
they cannot become independent after conditioning on other observed
variables.

\textbf{The EGA Procedure} The EGA framework follows a rigorous
three-step statistical sequence to identify the underlying
dimensionality of a dataset:

1. \textbf{Estimation:} The correlation matrix of observable variables
is estimated (e.g., tetrachoric or polychoric correlations for
categorical data).

2. \textbf{Regularization:} To prevent overfitting and the inclusion of
spurious correlations common in typical psychological sample sizes, the
\textbf{graphical LASSO (least absolute shrinkage and selection
operator)} is applied. The sparsity of the resulting network is
optimized by selecting a regularization parameter that minimizes the
\textbf{Extended Bayesian Information Criterion (EBIC)}.

3. \textbf{Community Detection:} The \textbf{walktrap algorithm}---a
random walk-based procedure---is used to identify dense subgraphs or
communities within the regularized partial correlation matrix. The
number of detected communities corresponds to the estimated number of
latent dimensions.

\textbf{Empirical Performance and Accuracy} Extensive simulation
research involving \textbf{32,000 datasets} across 64 conditions has
demonstrated that EGA is highly robust. While it performs comparably to
traditional methods in two-factor structures, EGA significantly
outperforms them in \textbf{four-factor structures with high
correlations (.70)} between dimensions. In these complex scenarios, EGA
was often the only technique capable of maintaining high accuracy,
reaching 100\% accuracy with a sample size of 5,000. Furthermore,
analysis of variance (ANOVA) indicates that EGA's accuracy is the
\textbf{least affected} by varying experimental conditions, such as
sample size and factor correlation, compared to traditional techniques.

\textbf{Additional Methodological Benefits} Beyond estimating the total
number of dimensions, EGA provides a distinct advantage by automatically
identifying \textbf{which specific items indicate each retrieved
dimension}. This dual functionality---dimensionality estimation and item
membership identification---provides a more comprehensive output for
construct validation than traditional factor-retention rules. Empirical
application to the \textbf{Inductive Reasoning Developmental Test
(IRDT)} further validates the framework, as EGA correctly identified
seven dimensions that matched the instrument's theoretical developmental
stages, while other methods like Parallel Analysis suggested only four

\subsection{This is an example for second level head---subsection
head}\label{subsec2}

\subsubsection{This is an example for third level head---subsubsection
head}\label{subsubsec2}

Sample body text. Sample body text. Sample body text. Sample body text.
Sample body text. Sample body text. Sample body text.

\section{Equations}\label{sec4}

Equations in \LaTeX~can either be inline or on-a-line by itself
(``display equations''). For inline equations use the \texttt{\$...\$}
commands. E.g.: The equation \(H\psi = E \psi\) is written via the
command \texttt{\$H\ \textbackslash{}psi\ =\ E\ \textbackslash{}psi\$}.

For display equations (with auto generated equation numbers) one can use
the equation or align environments:

\begin{equation}
\|\tilde{X}(k)\|^2 \leq\frac{\sum\limits_{i=1}^{p}\left\|\tilde{Y}_i(k)\right\|^2+\sum\limits_{j=1}^{q}\left\|\tilde{Z}_j(k)\right\|^2 }{p+q}.\label{eq1}
\end{equation}

where, \begin{align}
D_\mu &=  \partial_\mu - ig \frac{\lambda^a}{2} A^a_\mu \nonumber \\
F^a_{\mu\nu} &= \partial_\mu A^a_\nu - \partial_\nu A^a_\mu + g f^{abc} A^b_\mu A^a_\nu \label{eq2}
\end{align}

Notice the use of \texttt{\textbackslash{}nonumber} in the align
environment at the end of each line, except the last, so as not to
produce equation numbers on lines where no equation numbers are
required. The \texttt{\textbackslash{}label\{\}} command should only be
used at the last line of an align environment where
\texttt{\textbackslash{}nonumber} is not used.

\begin{equation}
Y_\infty = \left( \frac{m}{\textrm{GeV}} \right)^{-3}
    \left[ 1 + \frac{3 \ln(m/\textrm{GeV})}{15}
    + \frac{\ln(c_2/5)}{15} \right]
\end{equation}

The class file also supports the use of
\texttt{\textbackslash{}mathbb\{\}},
\texttt{\textbackslash{}mathscr\{\}} and
\texttt{\textbackslash{}mathcal\{\}} commands. As such
\texttt{\textbackslash{}mathbb\{R\}},
\texttt{\textbackslash{}mathscr\{R\}} and
\texttt{\textbackslash{}mathcal\{R\}} produces \(\mathbb{R}\),
\(\mathscr{R}\) and \(\mathcal{R}\) respectively (refer
Subsubsection~\ref{subsubsec2}).

\section{Tables}\label{sec5}

Tables can be inserted via the normal \texttt{knitr::kable()} function
or other table-generating packages.

\begin{table}

\caption{\label{tab:tab1}Caption text}
\centering
\begin{tabular}[t]{r|r}
\hline
temperature & pressure\\
\hline
0 & 0.0002\\
\hline
20 & 0.0012\\
\hline
40 & 0.0060\\
\hline
60 & 0.0300\\
\hline
80 & 0.0900\\
\hline
100 & 0.2700\\
\hline
\end{tabular}
\end{table}

Tables can also be inserted via the normal table and tabular
environment. To put footnotes inside tables you should use
\texttt{\textbackslash{}footnotetext{[}{]}\{...\}} tag. The footnote
appears just below the table itself (refer
Tables\textasciitilde{}\ref{tab1} and \ref{tab2}). For the corresponding
footnotemark use \texttt{\textbackslash{}footnotemark{[}...{]}}

\begin{table}[h]
\caption{Caption text}\label{tab1}%
\begin{tabular}{@{}llll@{}}
\toprule
Column 1 & Column 2  & Column 3 & Column 4\\
\midrule
row 1    & data 1   & data 2  & data 3  \\
row 2    & data 4   & data 5\footnotemark[1]  & data 6  \\
row 3    & data 7   & data 8  & data 9\footnotemark[2]  \\
\botrule
\end{tabular}
\footnotetext{Source: This is an example of table footnote. This is an example of table footnote.}
\footnotetext[1]{Example for a first table footnote. This is an example of table footnote.}
\footnotetext[2]{Example for a second table footnote. This is an example of table footnote.}
\end{table}

\noindent The input format for the above table is as follows:

\begin{verbatim}
\begin{table}[<placement-specifier>]
\caption{<table-caption>}\label{<table-label>}%
\begin{tabular}{@{}llll@{}}
\toprule
Column 1 & Column 2 & Column 3 & Column 4\\
\midrule
row 1 & data 1 & data 2  & data 3 \\
row 2 & data 4 & data 5\footnotemark[1] & data 6 \\
row 3 & data 7 & data 8  & data 9\footnotemark[2]\\
\botrule
\end{tabular}
\footnotetext{Source: This is an example of table footnote.
This is an example of table footnote.}
\footnotetext[1]{Example for a first table footnote.
This is an example of table footnote.}
\footnotetext[2]{Example for a second table footnote.
This is an example of table footnote.}
\end{table}
\end{verbatim}

\begin{table}[h]
\caption{Example of a lengthy table which is set to full textwidth}\label{tab2}
\begin{tabular*}{\textwidth}{@{\extracolsep\fill}lcccccc}
\toprule%
& \multicolumn{3}{@{}c@{}}{Element 1\footnotemark[1]} & \multicolumn{3}{@{}c@{}}{Element 2\footnotemark[2]} \\\cmidrule{2-4}\cmidrule{5-7}%
Project & Energy & $\sigma_{calc}$ & $\sigma_{expt}$ & Energy & $\sigma_{calc}$ & $\sigma_{expt}$ \\
\midrule
Element 3  & 990 A & 1168 & $1547\pm12$ & 780 A & 1166 & $1239\pm100$\\
Element 4  & 500 A & 961  & $922\pm10$  & 900 A & 1268 & $1092\pm40$\\
\botrule
\end{tabular*}
\footnotetext{Note: This is an example of table footnote. This is an example of table footnote this is an example of table footnote this is an example of~table footnote this is an example of table footnote.}
\footnotetext[1]{Example for a first table footnote.}
\footnotetext[2]{Example for a second table footnote.}
\end{table}

\vfill\eject

In case of double column layout, tables which do not fit in single
column width should be set to full text width. For this, you need to use
\texttt{\textbackslash{}begin\{table*\}} \texttt{...}
\texttt{\textbackslash{}end\{table*\}} instead of
\texttt{\textbackslash{}begin\{table\}} \texttt{...}
\texttt{\textbackslash{}end\{table\}} environment. Lengthy tables which
do not fit in textwidth should be set as rotated table. For this, you
need to use \texttt{\textbackslash{}begin\{sidewaystable\}} \texttt{...}
\texttt{\textbackslash{}end\{sidewaystable\}} instead of
\texttt{\textbackslash{}begin\{table*\}} \texttt{...}
\texttt{\textbackslash{}end\{table*\}} environment. This environment
puts tables rotated to single column width. For tables rotated to double
column width, use \texttt{\textbackslash{}begin\{sidewaystable*\}}
\texttt{...} \texttt{\textbackslash{}end\{sidewaystable*\}}.

\begin{sidewaystable}
\caption{Tables which are too long to fit, should be written using the "sidewaystable" environment as shown here}\label{tab3}
\begin{tabular*}{\textheight}{@{\extracolsep\fill}lcccccc}
\toprule%
& \multicolumn{3}{@{}c@{}}{Element 1\footnotemark[1]}& \multicolumn{3}{@{}c@{}}{Element\footnotemark[2]} \\\cmidrule{2-4}\cmidrule{5-7}%
Projectile & Energy & $\sigma_{calc}$ & $\sigma_{expt}$ & Energy & $\sigma_{calc}$ & $\sigma_{expt}$ \\
\midrule
Element 3 & 990 A & 1168 & $1547\pm12$ & 780 A & 1166 & $1239\pm100$ \\
Element 4 & 500 A & 961  & $922\pm10$  & 900 A & 1268 & $1092\pm40$ \\
Element 5 & 990 A & 1168 & $1547\pm12$ & 780 A & 1166 & $1239\pm100$ \\
Element 6 & 500 A & 961  & $922\pm10$  & 900 A & 1268 & $1092\pm40$ \\
\botrule
\end{tabular*}
\footnotetext{Note: This is an example of table footnote this is an example of table footnote this is an example of table footnote this is an example of~table footnote this is an example of table footnote.}
\footnotetext[1]{This is an example of table footnote.}
\end{sidewaystable}

\section{Figures}\label{sec6}

As per the \LaTeX~standards you need to use eps images for
\LaTeX~compilation and \texttt{pdf/jpg/png} images for \texttt{PDFLaTeX}
compilation. Use the \texttt{dev} knitr option to use the approrpate
format. This is one of the major difference between \LaTeX~and
\texttt{PDFLaTeX}. Each image should be from a single input .eps/vector
image file. Avoid using subfigures. The command for inserting images for
\LaTeX~and \texttt{PDFLaTeX} can be generalized. The package used to
insert images in \texttt{LaTeX/PDFLaTeX} is the graphicx package.
Figures can be inserted via the normal figure environment as shown in
the below example:

\begin{figure}
\includegraphics{Untitled_files/figure-latex/fig1-1} \caption{This is an example of a caption}\label{fig:fig1}
\end{figure}

\section{Algorithms, Program codes and Listings}\label{sec7}

Packages \texttt{algorithm}, \texttt{algorithmicx} and
\texttt{algpseudocode} are used for setting algorithms in \LaTeX~using
the format:

\begin{verbatim}
\begin{algorithm}
\caption{<alg-caption>}\label{<alg-label>}
\begin{algorithmic}[1]
. . .
\end{algorithmic}
\end{algorithm}
\end{verbatim}

You may refer above listed package documentations for more details
before setting \texttt{algorithm} environment. For program codes, the
``program'' package is required and the command to be used is
\texttt{\textbackslash{}begin\{program\}} \texttt{...}
\texttt{\textbackslash{}end\{program\}}. A fast exponentiation
procedure:

Similarly, for \texttt{listings}, use the \texttt{listings} package.
\texttt{\textbackslash{}begin\{lstlisting\}} \texttt{...}
\texttt{\textbackslash{}end\{lstlisting\}} is used to set environments
similar to \texttt{verbatim} environment. Refer to the
\texttt{lstlisting} package documentation for more details.

A fast exponentiation procedure:

\lstset{texcl=true,basicstyle=\small\sf,commentstyle=\small\rm,mathescape=true,escapeinside={(*}{*)}}
\begin{lstlisting}
begin
  for $i:=1$ to $10$ step $1$ do
      expt($2,i$);  
      newline() od                (*\textrm{Comments will be set flush to the right margin}*)
where
proc expt($x,n$) $\equiv$
  $z:=1$;
  do if $n=0$ then exit fi;
     do if odd($n$) then exit fi;                 
        comment: (*\textrm{This is a comment statement;}*)
        $n:=n/2$; $x:=x*x$ od;
     { $n>0$ };
     $n:=n-1$; $z:=z*x$ od;
  print($z$). 
end
\end{lstlisting}

\begin{algorithm}
\caption{Calculate $y = x^n$}\label{algo1}
\begin{algorithmic}[1]
\Require $n \geq 0 \vee x \neq 0$
\Ensure $y = x^n$ 
\State $y \Leftarrow 1$
\If{$n < 0$}\label{algln2}
        \State $X \Leftarrow 1 / x$
        \State $N \Leftarrow -n$
\Else
        \State $X \Leftarrow x$
        \State $N \Leftarrow n$
\EndIf
\While{$N \neq 0$}
        \If{$N$ is even}
            \State $X \Leftarrow X \times X$
            \State $N \Leftarrow N / 2$
        \Else[$N$ is odd]
            \State $y \Leftarrow y \times X$
            \State $N \Leftarrow N - 1$
        \EndIf
\EndWhile
\end{algorithmic}
\end{algorithm}

\begin{minipage}{\hsize}

\lstset{frame=single,framexleftmargin=-1pt,framexrightmargin=-17pt,framesep=12pt,linewidth=0.98\textwidth,language=pascal}

\begin{lstlisting}
for i:=maxint to 0 do begin \{ do nothing \} end; Write(`Case
insensitive'); Write(`Pascal keywords.');

\end{lstlisting}

\end{minipage}

\section{Cross referencing}\label{sec8}

Figures and tables are labeled with a prefix (fig or tab, respectively)
plus the chunk label. Other environments such as equation and align can
be labelled via the \texttt{\textbackslash{}label\{\#label\}} command
inside or just below the \texttt{\textbackslash{}caption\{\}} command.
You can then use the label for cross-reference. As an example, consider
the chunk label declared for Figure~\ref{fig:fig1} which is fig1. To
cross-reference it, use the command
\texttt{Figure\ \textbackslash{}ref\{fig:fig1\}}, for which it comes up
as ``Figure~\ref{fig:fig1}''.

To reference line numbers in an algorithm, consider the label declared
for the line number 2 of Algorithm~\ref{algo1} is
\texttt{\textbackslash{}label\{algln2\}}. To cross-reference it, use the
command \texttt{\textbackslash{}ref\{algln2\}} for which it comes up as
line~\ref{algln2} of Algorithm~\ref{algo1}.

\subsection{Details on reference citations}\label{subsec7}

For citations of references, use~\citet{bib1} or \citep{bib2}.

\section{Examples for theorem like environments}\label{sec10}

The documentclass for springer \texttt{sn-jnl.cls} contains 3 styling
that you can use to set new default for theorems and proofs type

\begin{description}
\item[\texttt{thmstyleone}]
Numbered, theorem head in bold font and theorem text in italic style
\item[\texttt{thmstyletwo}]
Numbered, theorem head in roman font and theorem text in italic style
\item[\texttt{thmstylethree}]
Numbered, theorem head in bold font and theorem text in roman style
\end{description}

For mathematics journals, theorem styles can be included as shown in the
following examples.

\begin{theorem}
Example theorem text. Example theorem text. Example theorem text.
Example theorem text. Example theorem text. Example theorem text.
Example theorem text. Example theorem text. Example theorem text.
Example theorem text. Example theorem text.

\end{theorem}

To add labels and subheadings, use LaTeX notation

\begin{theorem}[Theorem subhead]\label{thm1}
Example theorem text. Example theorem text. Example theorem text.
Example theorem text. Example theorem text. Example theorem text.
Example theorem text. Example theorem text. Example theorem text.
Example theorem text. Example theorem text.

\end{theorem}

Other environments are proposition, example, remark, definition, proof
and quote

Sample body text. Sample body text. Sample body text. Sample body text.
Sample body text. Sample body text. Sample body text. Sample body text.

\begin{proposition}
Example proposition text. Example proposition text. Example proposition
text. Example proposition text. Example proposition text. Example
proposition text. Example proposition text. Example proposition text.
Example proposition text. Example proposition text.

\end{proposition}

Sample body text. Sample body text. Sample body text. Sample body text.
Sample body text. Sample body text. Sample body text. Sample body text.

\begin{example}
Phasellus adipiscing semper elit. Proin fermentum massa ac quam. Sed
diam turpis, molestie vitae, placerat a, molestie nec, leo. Maecenas
lacinia. Nam ipsum ligula, eleifend at, accumsan nec, suscipit a, ipsum.
Morbi blandit ligula feugiat magna. Nunc eleifend consequat lorem.

\end{example}

Sample body text. Sample body text. Sample body text. Sample body text.
Sample body text. Sample body text. Sample body text. Sample body text.

\begin{remark}
Phasellus adipiscing semper elit. Proin fermentum massa ac quam. Sed
diam turpis, molestie vitae, placerat a, molestie nec, leo. Maecenas
lacinia. Nam ipsum ligula, eleifend at, accumsan nec, suscipit a, ipsum.
Morbi blandit ligula feugiat magna. Nunc eleifend consequat lorem.

\end{remark}

Sample body text. Sample body text. Sample body text. Sample body text.
Sample body text. Sample body text. Sample body text. Sample body text.

\begin{definition}[Definition sub head]
Example definition text. Example definition text. Example definition
text. Example definition text. Example definition text. Example
definition text. Example definition text. Example definition text.

\end{definition}

Additionally a predefined ``proof'' environment is available. This
prints a ``Proof'' head in italic font style and the ``body text'' in
roman font style with an open square at the end of each proof
environment.

\begin{proof}
Example for proof text. Example for proof text. Example for proof text.
Example for proof text. Example for proof text. Example for proof text.
Example for proof text. Example for proof text. Example for proof text.
Example for proof text.

\end{proof}

Sample body text. Sample body text. Sample body text. Sample body text.
Sample body text. Sample body text. Sample body text. Sample body text.

\section{Methods}\label{sec11}

Topical subheadings are allowed. Authors must ensure that their Methods
section includes adequate experimental and characterization data
necessary for others in the field to reproduce their work. Authors are
encouraged to include RIIDs where appropriate.

\textbf{Ethical approval declarations} (only required where applicable)
Any article reporting experiment/s carried out on
(i)\textasciitilde live vertebrate (or higher invertebrates),
(ii)\textasciitilde humans or (iii)\textasciitilde human samples must
include an unambiguous statement within the methods section that meets
the following requirements:

\begin{enumerate}
\def\labelenumi{\arabic{enumi}.}
\item
  Approval: a statement which confirms that all experimental protocols
  were approved by a named institutional and/or licensing committee.
  Please identify the approving body in the methods section
\item
  Accordance: a statement explicitly saying that the methods were
  carried out in accordance with the relevant guidelines and regulations
\item
  Informed consent (for experiments involving humans or human tissue
  samples): include a statement confirming that informed consent was
  obtained from all participants and/or their legal guardian/s
\end{enumerate}

If your manuscript includes potentially identifying patient/participant
information, or if it describes human transplantation research, or if it
reports results of a clinical trial then additional information will be
required. Please visit
(\url{https://www.nature.com/nature-research/editorial-policies}) for
Nature Portfolio journals,
(\url{https://www.springer.com/gp/authors-editors/journal-author/journal-author-helpdesk/publishing-ethics/14214})
for Springer Nature journals, or
(\href{https://www.biomedcentral.com/getpublished/editorial-policies\#ethics+and+consent}{https://www.biomedcentral.com/getpublished/editorial-policies\textbackslash\#ethics+and+consent})
for BMC.

\section{Discussion}\label{sec12}

Discussions should be brief and focused. In some disciplines use of
Discussion or `Conclusion' is interchangeable. It is not mandatory to
use both. Some journals prefer a section `Results and Discussion'
followed by a section `Conclusion'. Please refer to Journal-level
guidance for any specific requirements.

\section{Conclusion}\label{sec13}

Conclusions may be used to restate your hypothesis or research question,
restate your major findings, explain the relevance and the added value
of your work, highlight any limitations of your study, describe future
directions for research and recommendations.

In some disciplines use of Discussion or `Conclusion' is
interchangeable. It is not mandatory to use both. Please refer to
Journal-level guidance for any specific requirements.

\backmatter

\bmhead{Supplementary information}

If your article has accompanying supplementary file/s please state so
here.

Authors reporting data from electrophoretic gels and blots should supply
the full unprocessed scans for key as part of their Supplementary
information. This may be requested by the editorial team/s if it is
missing.

Please refer to Journal-level guidance for any specific requirements.

\bmhead{Acknowledgments}

Acknowledgments are not compulsory. Where included they should be brief.
Grant or contribution numbers may be acknowledged.

Please refer to Journal-level guidance for any specific requirements.

\section*{Declarations}\label{declarations}
\addcontentsline{toc}{section}{Declarations}

Some journals require declarations to be submitted in a standardised
format. Please check the Instructions for Authors of the journal to
which you are submitting to see if you need to complete this section. If
yes, your manuscript must contain the following sections under the
heading `Declarations':

\begin{itemize}
\tightlist
\item
  Funding
\item
  Conflict of interest/Competing interests (check journal-specific
  guidelines for which heading to use)
\item
  Ethics approval
\item
  Consent to participate
\item
  Consent for publication
\item
  Availability of data and materials
\item
  Code availability
\item
  Authors' contributions
\end{itemize}

\noindent If any of the sections are not relevant to your manuscript,
please include the heading and write `Not applicable' for that section.

\begin{flushleft}
Editorial Policies for:

\noindent Springer journals and proceedings:
\url{https://www.springer.com/gp/editorial-policies}

\noindent Nature Portfolio journals:
\url{https://www.nature.com/nature-research/editorial-policies}

\noindent \textit{Scientific Reports}:
\url{https://www.nature.com/srep/journal-policies/editorial-policies}

\noindent BMC journals:
\url{https://www.biomedcentral.com/getpublished/editorial-policies}

\end{flushleft}

\begin{appendices}

\section{Section title of first appendix}\label{secA1}

An appendix contains supplementary information that is not an essential
part of the text itself but which may be helpful in providing a more
comprehensive understanding of the research problem or it is information
that is too cumbersome to be included in the body of the paper.

For submissions to Nature Portfolio Journals please use the heading
``Extended Data''.

\end{appendices}

\bibliography{bibliography.bib}


\end{document}
