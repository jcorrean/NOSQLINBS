%Version 2.1 April 2023
% See section 11 of the User Manual for version history
%
%%%%%%%%%%%%%%%%%%%%%%%%%%%%%%%%%%%%%%%%%%%%%%%%%%%%%%%%%%%%%%%%%%%%%%
%%                                                                 %%
%% Please do not use \input{...} to include other tex files.       %%
%% Submit your LaTeX manuscript as one .tex document.              %%
%%                                                                 %%
%% All additional figures and files should be attached             %%
%% separately and not embedded in the \TeX\ document itself.       %%
%%                                                                 %%
%%%%%%%%%%%%%%%%%%%%%%%%%%%%%%%%%%%%%%%%%%%%%%%%%%%%%%%%%%%%%%%%%%%%%

\documentclass[sn-apa,pdflatex]{sn-jnl}

%%%% Standard Packages
%%<additional latex packages if required can be included here>

\usepackage{graphicx}%
\usepackage{multirow}%
\usepackage{amsmath,amssymb,amsfonts}%
\usepackage{amsthm}%
\usepackage{mathrsfs}%
\usepackage[title]{appendix}%
\usepackage{xcolor}%
\usepackage{textcomp}%
\usepackage{manyfoot}%
\usepackage{booktabs}%
\usepackage{algorithm}%
\usepackage{algorithmicx}%
\usepackage{algpseudocode}%
\usepackage{listings}%
%%%%

%%%%%=============================================================================%%%%
%%%%  Remarks: This template is provided to aid authors with the preparation
%%%%  of original research articles intended for submission to journals published
%%%%  by Springer Nature. The guidance has been prepared in partnership with
%%%%  production teams to conform to Springer Nature technical requirements.
%%%%  Editorial and presentation requirements differ among journal portfolios and
%%%%  research disciplines. You may find sections in this template are irrelevant
%%%%  to your work and are empowered to omit any such section if allowed by the
%%%%  journal you intend to submit to. The submission guidelines and policies
%%%%  of the journal take precedence. A detailed User Manual is available in the
%%%%  template package for technical guidance.
%%%%%=============================================================================%%%%

%% Per the spinger doc, new theorem styles can be included using built in style, 
%% but it seems the don't work so commented below
%\theoremstyle{thmstyleone}%
\newtheorem{theorem}{Theorem}%  meant for continuous numbers
%%\newtheorem{theorem}{Theorem}[section]% meant for sectionwise numbers
%% optional argument [theorem] produces theorem numbering sequence instead of independent numbers for Proposition
\newtheorem{proposition}[theorem]{Proposition}%
%%\newtheorem{proposition}{Proposition}% to get separate numbers for theorem and proposition etc.
%% \theoremstyle{thmstyletwo}%
\theoremstyle{remark}
\newtheorem{example}{Example}%
\newtheorem{remark}{Remark}%
%% \theoremstyle{thmstylethree}%
\theoremstyle{definition}
\newtheorem{definition}{Definition}%



\raggedbottom



% Pandoc syntax highlighting
\usepackage{color}
\usepackage{fancyvrb}
\newcommand{\VerbBar}{|}
\newcommand{\VERB}{\Verb[commandchars=\\\{\}]}
\DefineVerbatimEnvironment{Highlighting}{Verbatim}{commandchars=\\\{\}}
% Add ',fontsize=\small' for more characters per line
\usepackage{framed}
\definecolor{shadecolor}{RGB}{248,248,248}
\newenvironment{Shaded}{\begin{snugshade}}{\end{snugshade}}
\newcommand{\AlertTok}[1]{\textcolor[rgb]{0.94,0.16,0.16}{#1}}
\newcommand{\AnnotationTok}[1]{\textcolor[rgb]{0.56,0.35,0.01}{\textbf{\textit{#1}}}}
\newcommand{\AttributeTok}[1]{\textcolor[rgb]{0.13,0.29,0.53}{#1}}
\newcommand{\BaseNTok}[1]{\textcolor[rgb]{0.00,0.00,0.81}{#1}}
\newcommand{\BuiltInTok}[1]{#1}
\newcommand{\CharTok}[1]{\textcolor[rgb]{0.31,0.60,0.02}{#1}}
\newcommand{\CommentTok}[1]{\textcolor[rgb]{0.56,0.35,0.01}{\textit{#1}}}
\newcommand{\CommentVarTok}[1]{\textcolor[rgb]{0.56,0.35,0.01}{\textbf{\textit{#1}}}}
\newcommand{\ConstantTok}[1]{\textcolor[rgb]{0.56,0.35,0.01}{#1}}
\newcommand{\ControlFlowTok}[1]{\textcolor[rgb]{0.13,0.29,0.53}{\textbf{#1}}}
\newcommand{\DataTypeTok}[1]{\textcolor[rgb]{0.13,0.29,0.53}{#1}}
\newcommand{\DecValTok}[1]{\textcolor[rgb]{0.00,0.00,0.81}{#1}}
\newcommand{\DocumentationTok}[1]{\textcolor[rgb]{0.56,0.35,0.01}{\textbf{\textit{#1}}}}
\newcommand{\ErrorTok}[1]{\textcolor[rgb]{0.64,0.00,0.00}{\textbf{#1}}}
\newcommand{\ExtensionTok}[1]{#1}
\newcommand{\FloatTok}[1]{\textcolor[rgb]{0.00,0.00,0.81}{#1}}
\newcommand{\FunctionTok}[1]{\textcolor[rgb]{0.13,0.29,0.53}{\textbf{#1}}}
\newcommand{\ImportTok}[1]{#1}
\newcommand{\InformationTok}[1]{\textcolor[rgb]{0.56,0.35,0.01}{\textbf{\textit{#1}}}}
\newcommand{\KeywordTok}[1]{\textcolor[rgb]{0.13,0.29,0.53}{\textbf{#1}}}
\newcommand{\NormalTok}[1]{#1}
\newcommand{\OperatorTok}[1]{\textcolor[rgb]{0.81,0.36,0.00}{\textbf{#1}}}
\newcommand{\OtherTok}[1]{\textcolor[rgb]{0.56,0.35,0.01}{#1}}
\newcommand{\PreprocessorTok}[1]{\textcolor[rgb]{0.56,0.35,0.01}{\textit{#1}}}
\newcommand{\RegionMarkerTok}[1]{#1}
\newcommand{\SpecialCharTok}[1]{\textcolor[rgb]{0.81,0.36,0.00}{\textbf{#1}}}
\newcommand{\SpecialStringTok}[1]{\textcolor[rgb]{0.31,0.60,0.02}{#1}}
\newcommand{\StringTok}[1]{\textcolor[rgb]{0.31,0.60,0.02}{#1}}
\newcommand{\VariableTok}[1]{\textcolor[rgb]{0.00,0.00,0.00}{#1}}
\newcommand{\VerbatimStringTok}[1]{\textcolor[rgb]{0.31,0.60,0.02}{#1}}
\newcommand{\WarningTok}[1]{\textcolor[rgb]{0.56,0.35,0.01}{\textbf{\textit{#1}}}}

% tightlist command for lists without linebreak
\providecommand{\tightlist}{%
  \setlength{\itemsep}{0pt}\setlength{\parskip}{0pt}}





\begin{document}


\title[]{Labeled Property Graphs and Complex Psychological Data
Modeling}

%%=============================================================%%
%% Prefix	-> \pfx{Dr}
%% GivenName	-> \fnm{Joergen W.}
%% Particle	-> \spfx{van der} -> surname prefix
%% FamilyName	-> \sur{Ploeg}
%% Suffix	-> \sfx{IV}
%% NatureName	-> \tanm{Poet Laureate} -> Title after name
%% Degrees	-> \dgr{MSc, PhD}
%% \author*[1,2]{\pfx{Dr} \fnm{Joergen W.} \spfx{van der} \sur{Ploeg} \sfx{IV} \tanm{Poet Laureate}
%%                 \dgr{MSc, PhD}}\email{iauthor@gmail.com}
%%=============================================================%%

\author*[1,2]{\fnm{Juan} \spfx{C.} \sur{Correa} }\email{\href{mailto:j.correa.n@gmail.com}{\nolinkurl{j.correa.n@gmail.com}}}



  \affil*[1]{\orgdiv{Departmento de Estudios
Empresariales}, \orgname{Universidad
Iberoamericana}, \orgaddress{\city{Mexico
City}, \country{Mexico}, \postcode{1219}}}
  \affil*[2]{\orgdiv{Research and Development Unit}, \orgname{Critical
Centrality Institute}}

\abstract{Labeled property graphs offer a flexible framework for
modeling psychological phenomena as complex systems through graph data
structures such as multilevel designs, longitudinal assessments, and
dynamic patterns. This tutorial introduces graph databases and their
labeled property graphs as alternatives to relational databases,
illustrating how they enable efficient modeling of interconnected
behaviors. We provide a step-by-step guide for implementing property
graphs using Neo4j, including schema design, query patterns, and
integration with R/Python for analysis. Practical examples drawn from
open psychological datasets demonstrate performance and flexibility
gains over traditional relational approaches. Reproducible materials,
including code and data, are available via OSF to facilitate adoption
and replication.}

\keywords{Graph database, Network modeling, complex behavior}



\maketitle

\section{Introduction}\label{sec1}

In a recent article, \citet{soto2025} introduced relational databases
and illustrated their use by behavioral scientists through real-world
examples. Although relational databases with their standardized
``structured query language'' (SQL) remain the dominant paradigm in
research and industry, alternative approaches are gaining traction. The
so-called ``Not Only SQL'' (NoSQL) category includes database systems
that employ non-tabular data models, such as key-value pairs, documents,
wide columns, matrices, and graphs. Whereas relational databases store
structured data in tables (i.e., columns as variables, rows as cases,
and cells containing specific values), NoSQL systems accommodate
flexible formats that better support evolving and highly connected data,
which provides distinct advantages in psychological research.

In this article, I examine graphs as a database paradigm that deviates
from the traditional lenses of relational databases, where nodes and
edges (instead of tables and joins) represent the basic elements of any
behavior that can be represented as a network or a complex system.
Networks have a long history in mathematics as ``\emph{graph theory}''
\citep{Estrada2011}. In sociology and social sciences, graph theory is
known as ``social network analysis'' \citep{Wasserman1994}. In this
context, the term ``social network'' should not be confused with
Instagram or Facebook, as they are online platforms that do not
necessarily represent all aspects of social networks as an object of
study. Psychologists have leveraged this framework to analyze the
structure of psychopathology \citep{borsboom2013}, conduct bibliometric
analysis of cyberbehavior \citep{serafin2019}, estimate the correct
number of dimensions in psychological and educational instruments
\citep{golino2017}, or understand the measurement of organizational
climate \citep{menezes2021}. Even though these examples show the
versatility of social network analysis for psychological research, they
do not necessarily rely on graph database management systems and this
aspect is what differentiates this article. Here, I intend to illustrate
how the network lenses, when applied as a database management system,
unveils novel ways for modeling and analyzing behavioral data.

\section{Networks in a nutshell}\label{sec2}

\citet{Estrada2011} defines a network as a collection of points (called
nodes) joined together in pairs by lines (called arcs or edges) like
those depicted in Figure \hyperref[fig:F1]{1}. Despite this simplistic
definition, networks provide a powerful framework to model any type of
system from planets in a galaxy to neurons in the nervous system
\citep{Vazza2020}.

\begin{figure}[H]

{\centering \includegraphics[width=380px,]{../F1} 

}

\caption{A visual representation of four types of networks: A) non-directed unweighted network, B) non-directed weighted network, C) directed unweighted network, and D) directed weighted network.}\label{fig:F1}
\end{figure}

In behavioral sciences, networks have been used to understand the
mechanisms of team assembly and how these mechanisms determine
collaboration structure and team performance \citep{guimera2005}. Graphs
offer fundamental concepts for understanding how entities (nodes) and
their relationships (edges) form interconnected data structures. From a
data management viewpoint, the analysis of these data structures
requires tools that go beyond the rigid tabular constraints of
relational databases. As the concepts of graphs are thoroughly covered
in introductory texts \citep{Newman2010}, these will not be revisited
here. Instead, this article aims to illustrate how graph databases can
enrich the methodological toolbox of psychological researchers and
behavioral scientists, enabling analyses that embrace complexity,
dynamic relationships, and multi-level contingencies
\citep{Robinson2015}.

\section{Graph databases: A gentle
introduction}\label{graph-databases-a-gentle-introduction}

\citet{Robinson2015} define a graph database as a system that implements
\textbf{C}reate, \textbf{R}ead, \textbf{U}pdate, and \textbf{D}elete
(CRUD) operations on a graph data model, where entities are represented
as nodes and relationships as edges like the one depicted in Figure
\hyperref[fig:F2]{2}.

\begin{figure}[H]

{\centering \includegraphics[width=350px,height=180px,]{../F2} 

}

\caption{A visual representation of a graph database that combines persons, companies, and routes}\label{fig:F2}
\end{figure}

Unlike relational databases, which organize data in tables, graph
databases treat relationships as first-class elements rather than
secondary links between tables. This design enables efficient traversal
and pattern matching across highly connected data, making it ideal for
modeling complex networks such as behavioral contingencies or social
interactions. Nodes in Figure \hyperref[fig:F2]{2} refers to real-world
entities such as persons (i.e., Anna, Pam, Carlos, and John) companies
(i.e., Chevron and Rice University), and routes (i.e., I-10 and I-45).
Edges represent the relationship between pairs of nodes. Thus, John and
Pam work in Chevron but they commute distinct distances through
different routes. Carlos and Anna work in Rice University, they both
commute by the same route but they have to drive different distances.
Interestingly, the information of nodes and edges can be enriched with
attributes. These attributes also represent real-world characteristics
like the distance each person has to commute, their sex, or the sector
of the company they work for. This graph can become even more complex by
adding more properties on nodes. For example, the series of edges
labeled as \texttt{"COMMUTES\_BY"} that already have the property of
distance in miles can be enriched with other properties representing the
specific time-frame and the current travel time each person experiences
when using the road, allowing the integration of contextual
contingencies into individual commuting behavior. As nodes and edges can
represent distinct behaviors, contingencies, and interactions, graph
database management systems provide the unique opportunity to mingle
different kinds of objects accordingly. I will elaborate upon these
implications immediately.

\section{Why graph databases as psychological
method}\label{why-graph-databases-as-psychological-method}

A graph database model like the one depicted in Figure
\hyperref[fig:F2]{2} leverages the so-called ``labeled property graph.''
As a methodological tool it paves the way for incorporating complex
systems thinking in any discipline \citep{Krakauer2019}. In psychology
and behavioral sciences, complex systems thinking is not novel at all
\citep{Luce1999, Guastello2009}. However, a labeled property graph and
its elements have been largely overlooked. This gap coincides with
recent reviews that promote other methodologies as potential solutions
to increase intra and interdisciplinary interactions across subfields of
behavior analysis in both basic and applied settings \citep{Elcoro2023}.

A methodolgical implication embedded in graph databases is their
performance when handling network data structures \citep{Robinson2015}.
In relational databases, join-intensive queries slow down as datasets
grow, whereas graph databases maintain relatively stable performance
---even with millions of nodes and edges. From a traditional
perspective, the idea of working with huge datasets might seem
implausible for most psychological researchers. Nonetheless, big data
research is no longer outside professional considerations and that
explains why big data has been recently reviewed as a psychological
method \citep{vezzoli2023}. For example, previous works in network
science has shown how to address complex behavioral phenomena such as
the analysis of urban traffic at an individual level by tracking the
movements of hundreds of thousands of drivers every two hours
\citep{Gonzalez2008}. In cases like this, graph databases excel because
queries operate on localized portions of the graph rather than scanning
the entire dataset. If edges store information such as commuting
distances, queries can be designed to retrieve specific conditions
(e.g., individuals who commute less than 10 miles versus persons who
commute more than 20 miles). Consequently, execution time depends only
on the size of the subgraph traversed, not the overall graph size.

Another implication of graph databases is their flexibility. Behavioral
scientists aim to connect data in ways that reflect their knowledge and
expertise. This is why graph databases serve as the underlying
infrastructure for constructing ``knowledge graphs''
\citep{Barrasa2023}. With these knowledge graphs, researchers allow the
database to evolve alongside their understanding of the behavioral
phenomenon rather than being rigidly defined upfront, when knowledge is
most limited. This is particularly important when a behavioral
phenomenon lacks theoretical background or lacks replication
\citep{burgos2025}. Graph databases fulfill this need by providing a
flexible model that adapts to changing requirements and evidence.
Because graphs are inherently additive, new nodes, relationships,
labels, and subgraphs can be introduced. The modifications introduced to
the original graph do not imply a threat to existing queries or
application functionality. This flexibility minimizes the need for
exhaustive upfront modeling and lowers the frequency of costly
migrations (particularly for companies that hire behavior data
scientists in charge of analyzing customers observed behavior), thereby
reducing maintenance overhead in both academic and business settings.

A third implication of graph databases is the agility that they offer.
Modern graph databases enable smooth development and easy system
maintenance. Their schema-free design, combined with testable APIs and
query languages, allows controlled evolution of applications. While the
absence of rigid schemas means traditional governance mechanisms are
missing, this is not a drawback. On the contrary, it encourages more
transparent and actionable data governance. Typically, governance is
enforced programmatically through tests that validate data models,
queries, and business rules. This approach aligns well with agile and
test-driven development practices, making graph database applications
adaptable to changing business needs. In psychology, these aspects can
be vital for the so-called ``automated data collection techniques.'' For
example, mobile applications, such as ecological momentary assessment
tools, prompt users to report emotions or behaviors throughout the day,
while motion sensors and radio frequency identification tags can monitor
movement and location in homes or classrooms \citep{Bak2021}.

It is worth mentioning that relational databases acknowledge
relationships, but only during modeling, where they serve as join
mechanisms between tables. In graph databases, we often need to clarify
the meaning of relationships and even qualify their strength. These are
aspects that relational database management systems do not address
explicitly \citep{Robinson2015}. From this viewpoint, graph databases
demand both ontological and epistemological considerations like the ones
recently described in the literature of theoretical and philosophical
psychology \citep{burgos2025}. Roughly speaking, ontological
considerations relate to traditional questions of pure ontology like
what is it to be? what does exist? What is the nature of what exist? In
contrast, epistemological considerations pertains matters of evidence,
explanation, certainty and truth \citep{burgos2025}. As datasets grow
more complex and less uniform (provided new evidence is available),
relational data management systems like Microsoft Access, PostgreSQL, or
SQLite become burdened with large join tables, sparsely populated rows,
and extensive null-checking logic. Greater interconnectedness in
relational databases leads to more joins, which degrade performance and
complicate adaptation to evolving requirements. \citet{soto2025} has
highlighted some of these limitations as ``challenges to adoption.''
From the lenses of graph databases, however, these challenges are not
even necessary. I will elaborate upon this particular aspect in the next
section. As a wrap-up, labeled property graphs are not merely a
technical alternative to relational schemas; they represent a
methodological advance that aligns with psychology's need for flexible,
scalable, and conceptually transparent data models.

\section{Applying graph database in behavioral
research}\label{applying-graph-database-in-behavioral-research}

To illustrate the application of graph databases for behavioral
scientists, I revisit data of delivery time fulfillment from restaurants
offering food delivery in the city of Bogotá. The database is available
in a public data repository \citep{segura2019} which refers to an
associated research article \citep{correa2019a}. Behavioral scientists
who are familiar with structured datasets will find that this database
is in a coma-separated values (.csv) file, just like most tables used in
relational database management systems like Microsoft Access or SQLite.
Given the wide variety of graph database management systems in the
market (e.g., Neo4j, Microsoft Azure Cosmos, Aerospike, Amazon Web
Services Neptune, NebulaGraph, Memgraph, TigerGraph, Giraph), the rest
of this work relies on Neo4j \citep{Barrasa2023}.

Neo4j ranks as the leading freemium software in the segment of graph
database management systems and has been used in several industry
sectors including retail, finance, pharmaceutics, hospitality, and
electronic commerce, among others. Neo4j offers free and enterprise
editions, which can be installed on local or server environments
following official documentation. As the official documentation provides
helpful material for newcomers, downloading and installation details are
not necessary here, and the reader can consult specific details
elsewhere \citep{vanBruggen2014}.

\subsection{Instance creation and data
import}\label{instance-creation-and-data-import}

Instance creation is the first step in using graph databases with Neo4j.
This is a top-level operation for setting up a new database environment,
and involves allocating resources (memory, CPU, storage), defining a
database version, and setting up initial user credentials by clicking
one button. When creating a new instance, neo4j asks the user to provide
a name as part of the instance details. The user is asked to provide a
password of eight characters as a security check for further
interactions. The instance created is called ``OFD,'' an acronym for
``Online Food Delivery,'' and computational details can be found in our
public GitHub repository. By default any instance created in neo4j is in
``stopped'' mode so the user needs to start it by clicking one button.
The data import is straightforward using the import option depicted with
a red circle in the left panel of neo4j (see Figure
\hyperref[fig:F3]{3}).

\begin{figure}[H]

{\centering \includegraphics[width=380px,height=200px,]{../F3} 

}

\caption{The drag-and-drop import files option in the left panel of neo4j}\label{fig:F3}
\end{figure}

As described above, the database is a CSV file (newdata.csv) containing
19 variables. Delivery time fulfillment is recorded in seconds or
minutes (in the last two columns of the dataset). According to
\citet{segura2019}, this database was developed to evaluate the impact
of traffic conditions on key performance indicators for a sample of
restaurants offering food delivery services in Bogotá. It includes the
physical locations of both restaurants and customers, as well as
performance metrics and traffic data captured by the Google Maps API at
three time points during Saturday rush hours.

\subsection{Understanding the context}\label{understanding-the-context}

Traffic conditions in cities like Bogotá promote food delivery services
as an alternative to cooking or driving to get food \citep{correa2019a}.
As Bogotá has long suffered traffic congestion problems, local
governmental authorities have decided to impose a restriction program
called ``Pico y Placa'' \citep{montero2025}. This program was introduced
in August 1998 and has been modified several times looking to extend its
scope. For example, since July 2012, Pico y Placa affects the half of
residential and commercial vehicles every other day of the week
(excluding weekends) from 6:00 to 8:30 a.m. and then from 3:00 to 7:30
p.m, and buses, police cars, ambulances, fire trucks, government and
diplomatic vehicles, school buses and vans, and electric and hybrid
vehicles are exempt. To decide which half of the fleet is restricted in
any given day, the Pico y Placa follows an odd--even schedule based on
the last digit of the vehicle's license plate \citep{montero2025}.

In this context, \citet{correa2019a} reported that the data collected
from Google Maps API indicated that rush hours occur three times during
Saturdays: in the morning (between 8:00 and 10:00 a.m); around midday
(between 12:00 and 2:00 p.m); and in the evenings (between 6:00 and 8:00
p.m). Based on this information, \citet{correa2019a} classified the
typical traffic in three categories: ``free'' or ``green traffic'' (G),
``average'' or ``orange traffic'' (O), and ``heavy'' or ``red traffic'' (R).
By using letter triads they characterized the typical daily traffic with
sequences like ``G-G-G'' or ``R-O-R.'' Thus, for example, the sequence
``R-O-G'' means that the typical traffic changes from ``red'' in the
morning to ``orange'' at noon and ``green'' in the afternoon, describing
a place where traffic conditions improve as time passes.

\subsection{Sketching a graph database
model}\label{sketching-a-graph-database-model}

Sketching a graph database model is the next step after a successful
data import in Neo4j. Given the benefits of the labeled property graph,
an initial model can be modified multiple times. These modifications are
important because they reflect the analyst's expertise and highlight
critical nodes and relationships to be mapped. An initial graph database
model can consider the most elementary schema focused on the
relationship between restaurants and customers (Figure
\hyperref[fig:F4]{4}).

\begin{figure}[H]

{\centering \includegraphics[width=380px,height=100px,]{../F4} 

}

\caption{A visual representation of the initial schema about restaurants-customers relationship}\label{fig:F4}
\end{figure}

In this graph model, the node ``\texttt{Restaurant}'' has four hidden
properties (i.e., web, name, latitude, and longitude), the node
``\texttt{Customer}'' has two hidden properties (i.e., ClientLatitude,
ClientLongitude), and the relationship ``\texttt{SERVES}'' has one
property (i.e., distance). Despite its simple visual representation,
with this model we can examine several statistical aspects. For example,
by using cypher (i.e., the declarative, SQL-like query language for
property graphs in Neo4j) the following syntax provides the minimum, the
mean, and the maximum distance coverage from restaurants to customers'
location.

\begin{Shaded}
\begin{Highlighting}[]

\NormalTok{MATCH (:Restaurant){-}[s:SERVES]{-}\textgreater{}(:Customer)}
\NormalTok{RETURN round(avg(toFloat(s.Distance)), 1) AS avgDistanceMts,}
\NormalTok{       min(toFloat(s.Distance))           AS minDistanceMts,}
\NormalTok{       max(toFloat(s.Distance))           AS maxDistanceMts;}
\end{Highlighting}
\end{Shaded}

Despite the simplistic structure of the graph database model of Figure
\hyperref[fig:F4]{4}, its complexity and scope can easily grow as long
as the analyst intends to map other variable relationships. Before
moving on how to increase the complexity of this graph database model,
it is worth mentioning that it opens the possibility to examine the
\texttt{Restaurant-SERVES-Customer} resulting network with a single
cypher query like this one:

\begin{Shaded}
\begin{Highlighting}[]
\NormalTok{// Show the Restaurant–Customer network for distance \textless{}= 2000 meters}
\NormalTok{MATCH (r:Restaurant){-}[s:SERVES]{-}\textgreater{}(c:Customer)}
\NormalTok{WHERE toFloat(s.Distance) \textless{}= 2000}
\NormalTok{RETURN r, s, c;}
\end{Highlighting}
\end{Shaded}

With this cypher query, Neo4j produces a network visualization that
shows the graph data structure like the one depicted in Figure
\hyperref[fig:F5]{5}. Such a network-based data structure visualization
unlocks further analyses via Neo4j's built-in plugin called ``Graph Data
Science'' (GDS). This plugin provides extensive analytical capabilities
centered around graph algorithms, including nodes community detection,
centrality, similarity, path finding, and node embeddings, as well as
graph catalog procedures and machine learning pipelines designed to
support data science workflows for graphs. With GDS, all operations are
designed for massive scale and parallelization, with a custom and
general API tailored for graph-global processing, and highly optimized
compressed in-memory data structures.

\begin{figure}[H]

{\centering \includegraphics[width=380px,height=250px,]{../F5} 

}

\caption{A network-based visualization of the Restaurant-SERVES-Customer graph model}\label{fig:F5}
\end{figure}

From a methodological viewpoint, examining network-based data enables
the identification and testing of confounding factors in behavioral
non-experimental research designs. According to \citet{Kerlinger1986}, a
confounding factor (or hidden variable) refers to the mixing of the
statistical variance of one or more independent variables---typically
extraneous to the research purpose---with the independent variable or
variables of the research problem. For example, figure
\hyperref[fig:F5]{5} shows that the number of connections between
restaurants (green nodes) and customers (pink nodes) is uneven when
considering distances less than or equal to 2,000 meters (i.e., some
restaurants have more clients than others). If this pattern holds for
other distances, it is appropriate to analyze the restaurant-customer
relationships in the entire network. If this pattern does not hold, then
we can assume that restaurants' delivery time fulfillment might be
non-linearly related to the distance between restaurant and customer
locations. Non-linear relationships in network-like data structures
arise because connections between nodes follow a scale-free, power-law
distribution \citep{barabasi1999}.

The considerations described above entail further implications. As many
networks exhibit a power-law distribution---where the probability of
finding a highly connected node is relatively low compared to the high
probability of finding nodes with few connections---the development of
large networks is governed by ``self-organization'' which transcends the
properties of individual nodes and is therefore intrinsically linked to
the network structure. Self-organization is widely recognized in natural
sciences such as physics or biology, but in psychology is not a
mainstream framework \citep{correa2020}. Rougly speaking, emergence
refers to a property of the system that is not present by the individual
parts that belong to the system. The popular phrase ``the whole is
greater than the sum of its parts'' captures the idea of emergence,
which in psychology is largely attributed to the Gestalt school of
thought, but has been applied in ecological psychology \citep{Mace1977}
and information processing in cognitive psychology \citep{Hollis2009}.

\subsection{Increasing complexity in labeled property
graphs}\label{increasing-complexity-in-labeled-property-graphs}

As we mentioned before, labeled property graphs are highly flexible and
adaptable to new evidence and/or theoretical considerations. This
translates into redrawing the graph database model by adding more nodes
or edges. These modifications add not only more complexity on the
network structure, but increase the researcher's awareness on the
omission of other variables whose absence in the model may bias the
scrutiny of findings.

\backmatter

\bibliography{bibliography.bib}


\end{document}
