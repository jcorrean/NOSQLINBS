%Version 2.1 April 2023
% See section 11 of the User Manual for version history
%
%%%%%%%%%%%%%%%%%%%%%%%%%%%%%%%%%%%%%%%%%%%%%%%%%%%%%%%%%%%%%%%%%%%%%%
%%                                                                 %%
%% Please do not use \input{...} to include other tex files.       %%
%% Submit your LaTeX manuscript as one .tex document.              %%
%%                                                                 %%
%% All additional figures and files should be attached             %%
%% separately and not embedded in the \TeX\ document itself.       %%
%%                                                                 %%
%%%%%%%%%%%%%%%%%%%%%%%%%%%%%%%%%%%%%%%%%%%%%%%%%%%%%%%%%%%%%%%%%%%%%

\documentclass[sn-apa,pdflatex]{sn-jnl}

%%%% Standard Packages
%%<additional latex packages if required can be included here>

\usepackage{graphicx}%
\usepackage{multirow}%
\usepackage{amsmath,amssymb,amsfonts}%
\usepackage{amsthm}%
\usepackage{mathrsfs}%
\usepackage[title]{appendix}%
\usepackage{xcolor}%
\usepackage{textcomp}%
\usepackage{manyfoot}%
\usepackage{booktabs}%
\usepackage{algorithm}%
\usepackage{algorithmicx}%
\usepackage{algpseudocode}%
\usepackage{listings}%
%%%%

%%%%%=============================================================================%%%%
%%%%  Remarks: This template is provided to aid authors with the preparation
%%%%  of original research articles intended for submission to journals published
%%%%  by Springer Nature. The guidance has been prepared in partnership with
%%%%  production teams to conform to Springer Nature technical requirements.
%%%%  Editorial and presentation requirements differ among journal portfolios and
%%%%  research disciplines. You may find sections in this template are irrelevant
%%%%  to your work and are empowered to omit any such section if allowed by the
%%%%  journal you intend to submit to. The submission guidelines and policies
%%%%  of the journal take precedence. A detailed User Manual is available in the
%%%%  template package for technical guidance.
%%%%%=============================================================================%%%%

%% Per the spinger doc, new theorem styles can be included using built in style, 
%% but it seems the don't work so commented below
%\theoremstyle{thmstyleone}%
\newtheorem{theorem}{Theorem}%  meant for continuous numbers
%%\newtheorem{theorem}{Theorem}[section]% meant for sectionwise numbers
%% optional argument [theorem] produces theorem numbering sequence instead of independent numbers for Proposition
\newtheorem{proposition}[theorem]{Proposition}%
%%\newtheorem{proposition}{Proposition}% to get separate numbers for theorem and proposition etc.
%% \theoremstyle{thmstyletwo}%
\theoremstyle{remark}
\newtheorem{example}{Example}%
\newtheorem{remark}{Remark}%
%% \theoremstyle{thmstylethree}%
\theoremstyle{definition}
\newtheorem{definition}{Definition}%



\raggedbottom




% tightlist command for lists without linebreak
\providecommand{\tightlist}{%
  \setlength{\itemsep}{0pt}\setlength{\parskip}{0pt}}





\begin{document}


\title[Article Title runing]{Graph Databases as a Paradigm Shift in
Behavior Science}

%%=============================================================%%
%% Prefix	-> \pfx{Dr}
%% GivenName	-> \fnm{Joergen W.}
%% Particle	-> \spfx{van der} -> surname prefix
%% FamilyName	-> \sur{Ploeg}
%% Suffix	-> \sfx{IV}
%% NatureName	-> \tanm{Poet Laureate} -> Title after name
%% Degrees	-> \dgr{MSc, PhD}
%% \author*[1,2]{\pfx{Dr} \fnm{Joergen W.} \spfx{van der} \sur{Ploeg} \sfx{IV} \tanm{Poet Laureate}
%%                 \dgr{MSc, PhD}}\email{iauthor@gmail.com}
%%=============================================================%%

\author*[1,2]{\fnm{Juan} \spfx{C.} \sur{Correa} }\email{\href{mailto:j.correa.n@gmail.com}{\nolinkurl{j.correa.n@gmail.com}}}



  \affil*[1]{\orgdiv{Departmento de Estudios
Empresariales}, \orgname{Universidad
Iberoamericana}, \orgaddress{\city{Mexico
City}, \country{Mexico}, \postcode{1219}}}
  \affil*[2]{\orgdiv{Research and Development Unit}, \orgname{Critical
Centrality Institute}}

\abstract{Behavior scientists have multiple options for managing
research data. While relational databases offer robust tools for storage
and analysis, they impose structural constraints that limit the
representation of complex behavioral phenomena. This article argues that
graph databases provide a rigorous and conceptually richer alternative,
enabling the modeling of human behavior as an interconnected system
rather than a set of isolated variables. Beyond a technological shift,
adopting graph-based approaches invites a paradigm change in behavior
science---one that embraces complexity, dynamic relationships, and
multi-level contingencies. Practical implications are illustrated
through examples from clinical, consumer, and industrial/organizational
psychology.}

\keywords{Graph database, Network modeling, complex behavior}



\maketitle

\section{Introduction}\label{sec1}

In a recent paper, \citet{soto2025} introduced relational databases for
behavior science and used real-world examples to illustrate how
relational databases have been used by behavior scientists. Although
relational databases remain the dominant paradigm in research and
industry, alternative approaches are gaining traction. The so-called
`Not Only SQL' (NoSQL) category includes database systems that employ
non-tabular data models, such as key-value pairs, documents, wide
columns, matrices, and graphs. Whereas relational databases store
structured data in tables (i.e., columns as variables, rows as cases,
and cells containing specific values), NoSQL systems accommodate
flexible formats that better support evolving and highly connected data.

In this article, I examine graphs as a paradigm that deviates from the
traditional lenses of relational databases, where nodes and edges
(instead of tables and joins) represent the basic elements of any
behavior that can be represented as a network or complex system.
Networks have a long history in mathematics as ``\emph{graph theory}''
\citep{Estrada2011}. In sociology and social sciences, graph theory is
known as ``social network analysis'' \citep{Wasserman1994}. In this
context, the term ``social network'' should not be confused with online
platforms such as Facebook or Instagram, as they are technological
implementations that do not necessarily represent all aspects of social
networks as a discipline. Psychologists have leveraged this framework to
analyze the structure of psychopathology \citep{borsboom2013}, conduct
bibliometric analysis of cyberbehavior \citep{serafin2019}, estimate the
correct number of dimensions in psychological and educational
instruments \citep{golino2017}, or understand the measurement of
organizational climate \citep{menezes2021}.

\section{Network as a collection nodes and edges}\label{sec2}

One of the easiest way to grasp the idea of a network is by looking its
visual representation (see Figure \hyperref[fig:F1]{1}).

\begin{figure}[H]

{\centering \includegraphics[width=227px,]{../F1} 

}

\caption{A visual representation of four types of simple networks: non-directed unweighted network (upper left), non-directed weighted network (upper right), directed unweighted network (bottom left), and directed weighted network (bottom right).}\label{fig:F1}
\end{figure}

According to \citet{Estrada2011}, a network is a collection of points
(called nodes) joined together in pairs by lines (called arcs or edges).
Despite this simplistic definition, networks provide a powerful
framework to model any type of system from planets in a galaxy to
neurons in the nervous system \citep{Vazza2020}. In behavioral sciences,
networks have been used to understand the mechanisms of team assembly
and how these mechanisms determine collaboration structure and team
performance \citep{guimera2005}. Graphs offer fundamental concepts for
understanding how entities (nodes) and their relationships (edges) form
interconnected structures. From a data management viewpoint, the
analysis of these networks requires tools that go beyond the rigid
tabular constraints of relational databases. As the concepts of graphs
are thoroughly covered in introductory texts \citep{Newman2010}, these
will not be revisited here. Instead, this article aims to illustrate how
graph-based databases can enrich the methodological toolbox of behavior
scientists, enabling analyses that embrace complexity, dynamic
relationships, and multi-level contingencies \citep{Robinson2015}.

\section{Graph databases: A gentle
introduction}\label{graph-databases-a-gentle-introduction}

\citet{Robinson2015} define a graph database as a system that implements
\textbf{C}reate, \textbf{R}ead, \textbf{U}pdate, and \textbf{D}elete
(CRUD) operations on a graph data model, where entities are represented
as nodes and relationships as edges. Unlike relational databases, which
organize data in tables, graph databases treat relationships as
first-class elements rather than secondary links between tables. This
design enables efficient traversal and pattern matching across highly
connected data, making it ideal for modeling complex networks such as
behavioral contingencies or social interactions.

Figure \hyperref[fig:F2]{2} shows a graph with eight nodes and eight
edges. Nodes represent real-world entities such as persons (i.e., Ana,
Pam, Carlos, and John) companies (i.e., Chevron and Rice University),
and routes (i.e., I-10 and I-45). Edges represent the relationship
between pairs of nodes. Thus, John and Pam work in Chevron but they
commute distinct distances through different routes. Carlos and Ana work
in Rice University, they both commute by the same route but they have to
drive different distances. Interestingly, the information of nodes and
edges can be enriched with attributes. These attributes also represent
real-world characteristics like the distance each person has to commute,
their sex, or the sector of the company they work for. A graph database
like the one depicted in Figure \hyperref[fig:F2]{2} leverages the
``labeled property graph'' which has the following characteristics: 1)
it contains nodes and relationships, 2) Nodes contain properties
(key-value pairs), 3) Nodes can be labeled with one or more labels, 4)
Relationships are named and directed, and always have a start and end
node, 5) Relationships can also contain properties. The characteristics
of the labeled property graph provide a series of benefits for behavior
scientists.

\begin{figure}[H]

{\centering \includegraphics[width=320px,height=180px,]{../F2} 

}

\caption{A visual representation of a graph database that combines persons, companies, and routes}\label{fig:F2}
\end{figure}

\section{Benefits of graph databases}\label{benefits-of-graph-databases}

According to \citet{Robinson2015}, an advantage of graph databases is
the performance increase when dealing with connected data versus
relational databases working with tabular data. Compared with relational
databases, where join-intensive query performance deteriorates as the
dataset gets bigger, a graph database performance remains relatively
constant, even as the dataset grows with millions of records or
variables. Big data issues might not be present for experimental
behavior analysts who use to work with data coming from laboratory
experiments. Yet, applied behavior analysts can enjoy the the benefits
of graph databases. One example where graph databases can be
particularly helpful is when urban traffic in a city is under scrutiny
at an individual scale by tracking the movements of hundreds of
thousands of drivers in a city every two hours \citep{Gonzalez2008}. In
a case like the one just described, the outstanding performance of graph
databases is in its queries as they are localized to a portion of the
graph (e.g., if edges represent roads, the queries match the specific
edges to be consulted). As a result, the execution time for each query
is proportional only to the size of the part of the graph traversed to
satisfy that query, rather than the size of the overall graph.

Another advantage of graph databases has to do with their flexibility.
Behavior scientists aim to connect data in ways that reflect the
knowledge domain itself, allowing structure and schema to evolve
alongside their understanding of the problem rather than being rigidly
defined upfront, when knowledge is most limited. This is particularly
important when a behavioral phenomenon lacks theoretical background or
lacks replication \citep{burgos2025}. Graph databases fulfill this need
by providing a flexible model that adapts to changing requirements and
evidence. Because graphs are inherently additive, new nodes,
relationships, labels, and subgraphs can be introduced. The
modifications introduced to the original graph do not imply a threat to
existing queries or application functionality. This flexibility
minimizes the need for exhaustive upfront modeling and lowers the
frequency of costly migrations, thereby reducing maintenance overhead.

A third benefit of graph databases is the agility that they offer.
Modern graph databases enable smooth development and easy system
maintenance. Their schema-free design, combined with testable APIs and
query languages, allows controlled evolution of applications. While the
absence of rigid schemas means traditional governance mechanisms are
missing, this is not a drawback. On the contrary, it encourages more
transparent and actionable data governance. Typically, governance is
enforced programmatically through tests that validate data models,
queries, and business rules. This approach aligns well with agile and
test-driven development practices, making graph database--based
applications adaptable to changing business needs.

It is worth mentioning that relational databases acknowledge
relationships, but only during modeling, where they serve as join
mechanisms between tables. In graph databases, we often need to clarify
the meaning of relationships and even qualify their strength. These are
aspects that relational models cannot do \citep{Robinson2015}. From this
viewpoint, graph databases demand ontological considerations such as
those recently described for psychology and behavioral sciences
\citep{burgos2025}. As datasets grow more complex and less uniform,
relational data management systems like Microsoft Access, PostgreSQL, or
SQLite become burdened with large join tables, sparsely populated rows,
and extensive null-checking logic. Increased connectedness in relational
databases translates into more joins, which degrade performance and make
adapting the database to evolving requirements difficult. These and
other limitations of relational databases are highlighted as
``challenges to adoption'' \citep{soto2025}.

\section{Potential usages of graph databases in behavioral
research}\label{potential-usages-of-graph-databases-in-behavioral-research}

Even though network modeling has been used for several research purposes
like the one mentioned above, to my knowledge graph databases remain
underutilized by behavior scientists.

\backmatter

\bibliography{bibliography.bib}


\end{document}
