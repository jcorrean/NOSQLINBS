%Version 2.1 April 2023
% See section 11 of the User Manual for version history
%
%%%%%%%%%%%%%%%%%%%%%%%%%%%%%%%%%%%%%%%%%%%%%%%%%%%%%%%%%%%%%%%%%%%%%%
%%                                                                 %%
%% Please do not use \input{...} to include other tex files.       %%
%% Submit your LaTeX manuscript as one .tex document.              %%
%%                                                                 %%
%% All additional figures and files should be attached             %%
%% separately and not embedded in the \TeX\ document itself.       %%
%%                                                                 %%
%%%%%%%%%%%%%%%%%%%%%%%%%%%%%%%%%%%%%%%%%%%%%%%%%%%%%%%%%%%%%%%%%%%%%

\documentclass[sn-apa,pdflatex]{sn-jnl}

%%%% Standard Packages
%%<additional latex packages if required can be included here>

\usepackage{graphicx}%
\usepackage{multirow}%
\usepackage{amsmath,amssymb,amsfonts}%
\usepackage{amsthm}%
\usepackage{mathrsfs}%
\usepackage[title]{appendix}%
\usepackage{xcolor}%
\usepackage{textcomp}%
\usepackage{manyfoot}%
\usepackage{booktabs}%
\usepackage{algorithm}%
\usepackage{algorithmicx}%
\usepackage{algpseudocode}%
\usepackage{listings}%
%%%%

%%%%%=============================================================================%%%%
%%%%  Remarks: This template is provided to aid authors with the preparation
%%%%  of original research articles intended for submission to journals published
%%%%  by Springer Nature. The guidance has been prepared in partnership with
%%%%  production teams to conform to Springer Nature technical requirements.
%%%%  Editorial and presentation requirements differ among journal portfolios and
%%%%  research disciplines. You may find sections in this template are irrelevant
%%%%  to your work and are empowered to omit any such section if allowed by the
%%%%  journal you intend to submit to. The submission guidelines and policies
%%%%  of the journal take precedence. A detailed User Manual is available in the
%%%%  template package for technical guidance.
%%%%%=============================================================================%%%%

%% Per the spinger doc, new theorem styles can be included using built in style, 
%% but it seems the don't work so commented below
%\theoremstyle{thmstyleone}%
\newtheorem{theorem}{Theorem}%  meant for continuous numbers
%%\newtheorem{theorem}{Theorem}[section]% meant for sectionwise numbers
%% optional argument [theorem] produces theorem numbering sequence instead of independent numbers for Proposition
\newtheorem{proposition}[theorem]{Proposition}%
%%\newtheorem{proposition}{Proposition}% to get separate numbers for theorem and proposition etc.
%% \theoremstyle{thmstyletwo}%
\theoremstyle{remark}
\newtheorem{example}{Example}%
\newtheorem{remark}{Remark}%
%% \theoremstyle{thmstylethree}%
\theoremstyle{definition}
\newtheorem{definition}{Definition}%



\raggedbottom




% tightlist command for lists without linebreak
\providecommand{\tightlist}{%
  \setlength{\itemsep}{0pt}\setlength{\parskip}{0pt}}





\begin{document}


\title[Article Title runing]{Graphs as Not Only Relational Databases for
Behavior Science}

%%=============================================================%%
%% Prefix	-> \pfx{Dr}
%% GivenName	-> \fnm{Joergen W.}
%% Particle	-> \spfx{van der} -> surname prefix
%% FamilyName	-> \sur{Ploeg}
%% Suffix	-> \sfx{IV}
%% NatureName	-> \tanm{Poet Laureate} -> Title after name
%% Degrees	-> \dgr{MSc, PhD}
%% \author*[1,2]{\pfx{Dr} \fnm{Joergen W.} \spfx{van der} \sur{Ploeg} \sfx{IV} \tanm{Poet Laureate}
%%                 \dgr{MSc, PhD}}\email{iauthor@gmail.com}
%%=============================================================%%

\author*[1,2]{\fnm{Juan} \spfx{C.} \sur{Correa} }\email{\href{mailto:juan.correa@ibero.mx}{\nolinkurl{juan.correa@ibero.mx}}}



  \affil*[1]{\orgdiv{Departmento de Estudios
Empresariales}, \orgname{Universidad
Iberoamericana}, \orgaddress{\city{Mexico
City}, \country{Mexico}, \postcode{1219}}}
  \affil*[2]{\orgdiv{Research and Development Unit}, \orgname{Critical
Centrality Institute}}

\abstract{Behavior scientists have multiple options for managing
research data. While relational databases offer robust tools for storage
and analysis, they impose structural constraints that limit the
representation of complex behavioral phenomena. This article argues that
graph databases provide a rigorous and conceptually richer alternative,
enabling the modeling of human behavior as an interconnected system
rather than a set of isolated variables. Beyond a technological shift,
adopting graph-based approaches invites a paradigm change in behavior
science---one that embraces complexity, dynamic relationships, and
multi-level contingencies. Practical implications are illustrated
through examples from clinical, consumer, and industrial/organizational
psychology.}

\keywords{key, dictionary, word}


\pacs[JEL Classification]{D8, H51}
\pacs[MSC Classification]{35A01, 65L10}

\maketitle

\section{Introduction}\label{sec1}

In a recent paper, \citet{soto2025} introduced relational databases for
behavior science and used real-world examples to illustrate how
relational databases have been used by behavior scientists. Even though
relational databases represent the dominant paradigm inside and outside
academic research settings, other paradigms are gaining traction. The
so-called ``Not Only Relational Database'' encompasses a series of
database management systems that use single data structures to hold
information. There are several instances of single data structures such
as lists, key-value pairs, wide columns, documents, matrices, or graphs.

In this article, I examine graphs as a different paradigm from the
traditional paradigm of relational databases, where nodes and edges
(instead of tables and joins) represent the basic elements of any
behavior that can be represented as a network or complex system.
Networks have a long history in mathematics as ``\emph{graph theory}''
\citep{Estrada2011}. In sociology and social sciences, graph theory is
known as ``social network analysis'' \citep{Wasserman1994}, and
psychologists have leveraged this framework to analyze the structure of
psychopathology \citep{borsboom2013}, estimate the correct number of
dimensions in psychological and educational instruments
\citep{golino2017}, or understand the measurement of organizational
climate \citep{menezes2021}.

\section{Network as a collection nodes and edges}\label{sec2}

One of the easiest way to grasp the idea of a network is by looking its
visual representation (see Figure \hyperref[fig:F1]{1}). According to
\citet{Estrada2011}, a network is a collection of points (called nodes)
joined together in pairs by lines (called arcs or edges). Scientists can
model different kinds of networks from physical networks (e.g., flights
between airports) to biological networks (e.g., protein-protein
interactions), and social networks (e.g., who follows whom in LinkedIn
or X). In this context, the term ``social network'' should not be
confused with online platforms such as Facebook or Instagram, as they
are technological implementations that do not necessarily represent all
aspects of social networks as an academic discipline.

\begin{figure}[H]

{\centering \includegraphics[width=227px,]{../F1} 

}

\caption{A visual representation of four types of simple networks: non-directed unweighted network (upper left), non-directed weighted network (upper right), di- rected unweighted network (bottom left), and directed weighted network (bottom right).}\label{fig:F1}
\end{figure}

Graphs offer fundamental concepts for understanding how entities (nodes)
and their relationships (edges) form interconnected structures. Network
structures underpin a wide range of behavioral phenomena ---from disease
transmission and social clustering to the spread of information and
misinformation. As we have mentioned above, the recognition of these
patterns requires tools that go beyond the rigid tabular constraints of
relational databases. As the concepts of graphs are thoroughly covered
in introductory texts \citep{Newman2010}, these will not be revisited
here. Instead, this article aims to illustrate how graph-based databases
can enrich the methodological toolbox of behavior scientists, enabling
analyses that embrace complexity, dynamic relationships, and multi-level
contingencies \citep{Robinson2015}.

\section{Graph databases: A gentle
introduction}\label{graph-databases-a-gentle-introduction}

\citet{Robinson2015} define a graph database as a system that implements
\textbf{C}reate, \textbf{R}ead, \textbf{U}pdate, and \textbf{D}elete
(CRUD) operations on a graph data model, where entities are represented
as nodes and relationships as edges. Unlike relational databases, which
organize data in tables (columns as variables, rows as cases, and cells
containing specific information for each case-variable combination),
graph databases treat relationships as first-class elements rather than
secondary links between tables. This design enables efficient traversal
and pattern matching across highly connected data, making it ideal for
modeling complex networks such as behavioral contingencies or social
interactions.

In Figure \hyperref[fig:F2]{2} we can see a graph of three X users
(i.e., Tom, Paul, and Anna) and three messages (i.e., 99, 100, and 101).
Tom is followed by Anna and Paul, Tom follows both Anna and Paul, but
Paul does not follow Anna, who has a string of messages. Her most recent
message can be found by following a relationship marked CURRENT, and the
PREVIOUS relationships then create Anna's timeline.

\begin{figure}[H]

{\centering \includegraphics[width=120px,height=180px,]{../F2} 

}

\caption{A visual representation of a graph database}\label{fig:F2}
\end{figure}

A graph database like the one depicted in Figure \hyperref[fig:F2]{2}
leverages the ``labeled property graph'' which has the following
characteristics: 1) it contains nodes and relationships, 2) Nodes
contain properties (key-value pairs), 3) Nodes can be labeled with one
or more labels, 4) Relationships are named and directed, and always have
a start and end node, 5) Relationships can also contain properties.

\subsection{Benefits of graph
databases.}\label{benefits-of-graph-databases.}

According to \citet{Robinson2015}, a compelling reason for choosing a
graph database is the sheer performance increase when dealing with
connected data versus relational databases. In contrast to relational
databases, where join-intensive query performance deteriorates as the
dataset gets bigger, with a graph database performance remain relatively
constant, even as the dataset grows with millions of records. This is
because queries are localized to a portion of the graph (the specific
edge(s) to be consulted). As a result, the execution time for each query
is proportional only to the size of the part of the graph traversed to
satisfy that query, rather than the size of the overall graph.

\backmatter

\bibliography{bibliography.bib}


\end{document}
